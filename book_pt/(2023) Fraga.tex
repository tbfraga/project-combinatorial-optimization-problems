\documentclass{book}
%\usepackage[brazilian]{babel}
%\usepackage[utf8]{inputenc}
%\usepackage[T1]{fontenc}
%\usepackage{multirow}

\usepackage{amsmath}
\usepackage{amsfonts}
%\usepackage{graphicx}
%\graphicspath{ {Figures/} }
%\usepackage{float}
%\usepackage[dvipsnames]{xcolor}
%\usepackage{tcolorbox}

%\usepackage{natbib,stfloats}

%\tcbuselibrary{skins,breakable}
%\usetikzlibrary{shadings,shadows}
%\usepackage{rotating}

%\usepackage{tikz}
%\usetikzlibrary{snakes}
%\usetikzlibrary{patterns}

%\usepackage{colortbl}
%\newcolumntype{a}{>{\columncolor{yellow}}l}
%\newcolumntype{b}{>{\columncolor{blue}}l}
%\newcolumntype{d}{>{\columncolor{green}}l}

%\usepackage{fancyhdr}
%\pagestyle{fancy}

%\fancyhead{}
%\fancyhead[RO]{\textsl{\rightmark}} 
%\fancyhead[LE]{\textsl{\leftmark}} 

%\fancyfoot{} 
%\fancyfoot[L]{T.B. Fraga (2023). Combinatorial Optimization Problems.}
%\fancyfoot[R]{\thepage}

%\definecolor{cambridgeblue}{rgb}{0.64, 0.76, 0.68}
%\definecolor{camel}{rgb}{0.76, 0.6, 0.42}
%\definecolor{camouflagegreen}{rgb}{0.47, 0.53, 0.42}
%\definecolor{desertsand}{rgb}{0.93, 0.79, 0.69}

\newenvironment{exerciseblock}[1]{%
    \tcolorbox[beamer,%
    noparskip,breakable,
    colback=camel!20,colframe=camouflagegreen,%
    colbacklower=camouflagegreen!75!camel!20,%
    title=#1]}%
    {\endtcolorbox}

\newenvironment{exempleblock}[1]{%
    \tcolorbox[beamer,%
    noparskip,breakable,
    colback=camel!20,colframe=Brown!70,%
    colbacklower=camouflagegreen!75!camel!20,%
    title=#1]}%
    {\endtcolorbox}

\newenvironment{algorithmblock}[1]{%
    \tcolorbox[beamer,%
    noparskip,breakable,
    colback=camel!20,colframe=Brown!40,%
    colbacklower=camouflagegreen!75!camel!20,%
    title=#1]}%
    {\endtcolorbox}

\newenvironment{dedication}
  {\clearpage           % we want a new page
   \thispagestyle{empty}% no header and footer
   \vspace*{\stretch{1}}% some space at the top 
   \itshape             % the text is in italics
   \raggedleft          % flush to the right margin
  }
  {\par % end the paragraph
   \vspace{\stretch{3}} % space at bottom is three times that at the top
   \clearpage           % finish off the page
  }

% para construção de diagramas

%\tikzstyle{place}=[circle,draw=camouflagegreen!80,fill=desertsand!20,thick]
%\tikzstyle{Ret}=[rectangle,draw=camouflagegreen!80,fill=desertsand!20,thick]
%\tikzstyle{RoundRet}=[rectangle,draw=camouflagegreen!80,fill=desertsand!20,thick, rounded corners]

\begin{document}

Livro em processo de elaboração. \\

Título: Combinatorial Optimization Problems\\

Autora: Tatiana Balbi Fraga. \\

Profa. do Núcleo de Tecnologia da UFPE. \\

Caso tenha interesse em participar da elaboração deste livro, enviar e-mail para: tatiana.balbi@ufpe.br.\\

O mesmo endereço de e-mail poderá ser utilizado para críticas e sugestões. Desde já agradeço por qualquer contribuição. \\

O conteúdo apresentado neste livro estará sendo modificado ao longo de sua elaboração, caso o mesmo seja consultado para elaboração de algum trabalho, favor citar como referência o título, a autora, a data da consulta e o endereço da versão consultada no github. \\

\chapter{Problemas de Otimização Combinatória}

Há três pontos chaves na construção de um algoritmo para solução de problemas de otimização combinatória que são: representação da solução; criação de solução inicial / nova solução dentro do espaço viável; e mecanismo de conversão. \\

Neste livro pretendo apresentar uma revisão da literatura focada nestes três tópicos, iniciando a abordagem sobre problemas de designação, depois passando pelos problemas de balanceamento, roteamento, e escalonamento até finalmente chegar aos complexos problemas mistos. \\

Com vista a este propósito, o livro será dividido em seis capítulos, sendo os dois primeiros capítulos voltados à apresentação de importantes classes de problemas de otimização combinatória (problemas de otimização combinatória e modelagem matemática), os três capítulos seguintes voltados aos três pontos chaves anteriormente discutidos (representação de soluções, construção de soluções viáveis e mecanismos de busca), e o último capítulo será utilizado para a apresentação de algoritmos conhecidos aplicados aos diferentes problemas e uma comparação entre tais algoritmos. \\

Esse livro partiu da ideia da criação de um solver capaz de solucionar problemas de otimização combinatória de diferentes classes. \\

Esse solver terá como princípio a identificação da natureza do problema, possivelmente através de um algoritmo de rede neural, e posterior solução focando nos três pontos chaves anteriormente discutidos. \\

Assim sendo, este livro será construído em conjunto com o solver (em C++), o qual poderá ser encontrado na pasta solver deste mesmo diretório. \\

Este projeto estará sendo desenvolvido em conjunto com outros projetos, de forma que não será possível desenvolvê-lo muito rapidamente... \\

Espero que essa ideia resulte em um material muito útil, relevante e interessante.

\section{Problema de dimensionamento do tempo de processamento de lotes de produção}

O problema de dimensionamento do tempo de processamento de lotes de produção surge quando um conjunto de produtos distintos são processados simultaneamente em um mesmo lote de produção, sendo a quantidade produzida de cada produto diretamente proporcional ao tempo de processamento, contudo, com taxa de produção (quantidade/(unidade de tempo)) diferente para cada produto. Neste problema consideramos que há uma quantidade máxima permitida para a produção dos produtos do lote, definida tanto individualmente como para o conjunto, já que existe uma quantida máxima de produção para cada produto e uma quantida máxima de produção definida para o conjunto. A quantidade máxima de produção de cada produto é definida de acordo com a demanda do produto. Contudo é ainda possível estocar os produtos e/ou enviá-los para as lojas de fábrica. Tanto no caso do estoque em fábrica quanto no caso das lojas de fábrica, existe um limite para estocagem/envio de cada produto e um limite de estocagem/envio para o conjunto de produtos do lote. O problema consiste em definir o maior tempo de processamento do lote viável considerando um período fixo de planejamento e demanda variável dentro deste período. \\

\subsection{Exemplo}

Determinada máquina deve processar um lote contendo 2 produtos distintos: A e B. \\

A taxa de produção de A é de 60 g/min enquanto que a taxa de produção de B é de 40 g/min. \\

A fábrica tem estoque livre para no máximo 3000 g de qualquer produto, sendo que, de acordo com o estoque máximo permitido para cada produto, poderá ser estoqcado na fábrica mais 3000 g do produto A e 2000 g do produto B. \\

Existe uma demanda de 1000 g do produto A e 500 g do produto B para entrega no primeiro e segundo dia de planejamento, respectivamente. \\

A fábrica tem um outlet que tem espaço livre em estoque de 1000 g, poderendo receber no máximo 600 g de cada produto. \\

Um tempo máximo de 300 minutos desta máquina pode ser atribuído para processamento deste lote. \\

Qual é o tempo de processamento deste lote ? \\

\section{Modelagem matemática}

sendo: \\

$\textrm{UD}_i$ a demanda referente ao produto $i$; \\

$\textrm{I}$ a quantidade máxima permitida para estocagem em fábrica de todos os produtos; \\

$\textrm{UI}_i$ a quantidade máxima permitida para estocagem em fábrica do produto $i$; \\

$\textrm{O}$ a quantidade máxima permitida para envio de todos os produtos para as lojas de fábrica; \\

$\textrm{UO}_i$ a quantidade máxima do produto $i$ que pode ser enviada para as lojas de fábrica; \\ 

$\textrm{p}_i$ a taxa de produção do produto $i$; \\

$\textrm{Z}$ tempo limite para processamento do lote; \\

queremos definir tempo máximo de processamento do lote, $T$. \\

assim sendo, considerando que: \\

$P_i$ a quantidade produzida do produto $i$ (unidades); \\

$P_i - \textrm{p}_i * T  = 0 \quad \forall i$ \\

$P_i - D_i - O_i - I_i = 0 \quad \forall i$ \\

$D_i$ a quantidade entregue para demanda do produto $i$; \\

$D_i \leq \textrm{UD}_i \quad \forall i$ \\

$O_i$ quantidade do produto $i$ enviado para as lojas de fábrica; \\

$O_i \leq \textrm{UO}_i \quad \forall i$\\

$\sum_i{O_i} \leq \textrm{O} \quad \forall i$ \\

$I_i$ a quantidade do produto $i$ que será estocado em fábrica; \\

$I_i \leq \textrm{UI}_i \quad \forall i$ \\

$\sum_i{I_i} \leq \textrm{I} \quad \forall i$ \\

e, por ultimo temos as restrições relacionadas às variáveis de decisão: \\

$P_i, D_i, O_i, I_i \in \textrm{Z}^* \quad \forall i $ \\


Portanto temos o problema de otimização: \\


$max \quad T$ \\

$s.t.$ \\

$P_i - \textrm{p}_i * T  = 0 \quad \forall i$ \\

$P_i - D_i - O_i - I_i = 0 \quad \forall i$ \\

$D_i \leq \textrm{UD}_i \quad \forall i$ \\

$O_i \leq \textrm{UO}_i \quad \forall i$ \\

$\sum_i{O_i} \leq \textrm{O} \quad \forall i$ \\

$I_i \leq \textrm{UI}_i \quad \forall i$ \\

$\sum_i{I_i} \leq \textrm{I} \quad \forall i$ \\

$T \leq \textrm{Z}$ \\

$P_i, D_i, O_i, I_i \in  \mathbb{Z}^+ \quad \forall i$


\section{Solução analítica e matemática}

1) Na verdade não há diferença entre a quantidade que será estocada e a quantidade que será enviada para as lojas de fábrica. Assim, podemos considerar o estoque na fábrica nas lojas de fábrica como sendo um único estoque. \\

$E_i = O_i + I_i \quad \forall i $ \\

$E_i \leq \textrm{UO}_i + \textrm{UI}_i \quad \forall i$ \\

$\sum_i{E_i} \leq \textrm{O} + \textrm{I} \quad \forall i$ \\

2) Portanto podemos resolver o problema de uma forma recursiva:

 - primeiro resolvemos um problema menor, considerando apenas a quantidade necessária para atender à demanda
 
 		-- encontramos máximo T' que atende o conjunto de equações \\
 		
 		 $D_i \leq \textrm{UD}_i \quad \forall i$ \\
 		
 		-- então calculamos a sobra (quantidade da demanda não atendida) \\
 		
 		$S_i = \textrm{UD}_i - D_i \quad \forall i$ \\
 
 - em seguida encontramos máximo T'' resolvendo as equações \\
 
 $E_i - S_i \leq \textrm{UO}_i + \textrm{UI}_i \quad \forall i$ \\

$\sum_i{(E_i- S_i)} \leq \textrm{O} + \textrm{I} \quad \forall i$ \\

%\bibliographystyle{acm}
%\bibliography{Bibliography/BibFile}

%\begin{thebibliography}{12}

%\bibitem[\protect\citeauthoryear{Ballou}{2001}]{Ballou2001}
%Ballou, R.H. (2001).{\it Gerenciamento da Cadeia de Suprimentos: Planejamento, Organização e Logística Empresarial}, 4. ed., Porto Alegre: Bookman.

%\bibitem[\protect\citeauthoryear{Boylan et al.}{2008}]{BoylanEtAl2008}
%Boylan, J.E., Syntetos, A.A., e Karakostas, G.C. (2008). 'Classification for forecasting and stock control:a case %study'. {\it Journal of the Operational Research Society}, Vol. 59, pp. 473--481.

%\end{thebibliography}

\end{document}
