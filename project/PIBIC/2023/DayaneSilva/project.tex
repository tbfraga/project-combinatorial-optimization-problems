\documentclass[11pt,a4paper]{article}
% This text is inserted in the beginning of all
% LaTex and Tex files I create.
%
% File created: Fri Jul 07 2023
% File name:    report_template.tex
% Path:         /home/name/Classes/AMPIII/Template/
%
% Tatiana Balbi Fraga
% Jul, 2023
%

\usepackage[english]{babel}
\usepackage[utf8]{inputenc}

%recommended by fancyhdr package
\setlength{\headheight}{14.49998pt}
\addtolength{\topmargin}{-2.49998pt}

% include a minimal set of useful packages
\usepackage{graphicx}
\usepackage{amsfonts} 
\usepackage{amssymb}
\usepackage{amsmath}
\usepackage[a4paper,left=3.0cm,top=3.0cm,right=2.0cm,bottom=2.0cm]{geometry}
\usepackage{lastpage}
\usepackage{fancyhdr}
\usepackage{natbib,stfloats}
\usepackage[parfill]{parskip}
\usepackage{multirow}
\usepackage{helvet}

% PUT YOUR TITLE AND NAME HERE
\newcommand{\titlestr}{Planejamento do transporte compartilhado em vans \\ Projeto PIBIC}
\newcommand{\shorttitlestr}{Planejamento do transporte compartilhado em vans}
\newcommand{\authorstr}{Profa. Tatiana Balbi Fraga} % INSERT YOUR NAME(S)

\begin{document}
%%%%%%%%%%%%%%%%%%%%%%%%%%%%%%555
% title page
\begin{titlepage}
  \centering
  
  {\LARGE \titlestr \par}

  \vspace{1cm}
  {\Large \authorstr \par}

  {\bf Aluna: Dayane Eduarda da Silva}

  \vspace{1cm}
  \today     % PUT YOUR DATE HERE

  \vspace{1.3cm}
  Projeto submetido ao
  {\bf Edital PROPESQI no 02/2022}
  da Pró-Reitoria de Pesquisa e Inovação
  da Universidade Federal de Pernambuco

  \includegraphics[width=0.25\textwidth]{logos/ufpe_logo.png}

  \vspace{1.8cm}
  \flushleft
  {\bf Departamento de Engenharia de Produção} \\
  {\bf Núcleo de Tecnologia} \\
  {\bf Centro Acadêmico do Agreste} \\

 \vspace{5mm} {\footnotesize Este projeto é parte do projeto de pesquisa registrado na Propesqi intitulado 'Análise e Modelagem de Problemas de Otimização Contínua e Combinatória' \citep{Fraga2018}. Ao submeter este projeto eu declaro que o mesmo é de minha autoria e que todas as referências consultadas estão claramente citadas. O projeto está embasado no projeto desenvolvido pela aluna Dayane Eduarda da Silva (disponível no github através do site: www.github.com/tbfraga/project-combinatorial-optimization-problems/library), sob minha orientação, na disciplina 'Português Instrumental e Metodologia Científica'. Contudo o projeto que aqui apresento foi elaborado após extenso trabalho de pesquisa realizado por mim (incluindo levantamento bibliogŕafico e proposição da metodologia a ser aplicada).\par}
    
  \vspace{2mm} {\footnotesize O texto do projeto (atualizado), estará disponível em forma de artigo no site MadScientistsDiary: https://github.com/tbfraga/gamos-madScientistsDiary-, portanto qualquer utilização, alteração, distribuição, replicação deste projeto deve estar em conformidade com a Licença Pública Internacional Creative Commons Atribuição-NãoComercial-SemDerivativos 4.0 e qualquer outro projeto desenvolvido com base neste projeto deve citá-lo como referência respeitando os direitos de autoria.\par}
    
  \vfill
\end{titlepage}

% put headings on each page
\pagestyle{fancy}
\fancyhf{}
\rhead{\shorttitlestr}
\lhead{\authorstr}
\rfoot{Page \thepage\ of \pageref{LastPage}}
\renewcommand{\headrulewidth}{1pt}


%%%%%%%%%%%%%%%%%%%%%%%%%%%%%%555
% main report
\clearpage
\section{Introdução}

Transporte compartilhado é um sistema de transporte no qual um grupo de pessoas compartilham um mesmo veículo. O compartilhamento pode ser feito tanto de forma simultânea (\emph{e.g}, compartilhamento de carros), quanto de forma temporal (\emph{e.g}, compartilhamento de bicicletas) \citep{Wikipedia}. De acordo com o \cite{SMC} há varios tipos de transporte compartilhado sendo um destes o \emph{Vanpooling}. Citanto o site "o \emph{Vanpooling} é um serviço baseado em assinatura, onde um motorista oferece passeios pré-distribuídos a 3 a 15 passageiros com quem ele compartilha uma origem e destino. Geralmente administrado por um órgão público, distrito comercial ou local de trabalho, programas de vanpooling normalmente arrendam e disponibilizam os veículos (minivans ou vans de passageiros) e os participantes compartilham uma taxa mensal que cobre o custo, o seguro, a manutenção e o gás do veículo - geralmente muito menor do que o custo de fazer a mesma viagem diária em um veículo pessoal."

A literatura científica apresenta alguns trabalhos que abordam o tema \emph{Vanpooling}. \cite{Nassar1986} propôs um modelo matemático para simulação de um serviço de transporte do tipo \emph{Vanpooling}. Após realizar avaliações do problema e simulações com o modelo proposto, o autor concluiu que o transporte compartilhado em vãs (\emph{Vanpool}) apresenta como vantagens a redução dos problemas de estacionamento, dos custos de transporte, e dos atrasos e abcenteísmos. \cite{KaanOlinick2013} apresentaram um modelo matemático para um problema de otimização do transporte \emph{Vanpool}, denomidado pelo autores como MCVAM. O MCVAN considera restrições de capacidade e relacionadas ao custo e ao tempo de transporte e tem como objetivo minimizar o custo total do transporte de um grupo de pessoas que devem ser transportadas.  Já \cite{DitmoreDeming2018} realizaram um estudo para determinar se passageiros que utilizam o \emph{Vanpool} para realizar o delocamento casa-trabalho-casa experimentam um nível diferenciado de estresse quando comparado ao transporte em veículos com um único ocupante. Como resultado, os autores puderam concluir que o \emph{Vanpool} ajudou a diminuir o estresse experimentado pelos trabalhadores durante os dias de trabalho.

Neste projeto será feito o estudo de um serviço de transporte do tipo \emph{Vanpooling} buscando um melhor planejamento para este serviço com foco na satisfação das pessoas transportadas.

\section{Objetivos}

Como objetivo geral, este projeto buscará melhorar o planejamento de um serviço de \emph{Vanpooling} utilizado para transporte dos alunos do CAA-UFPE.

Buscando atender esté proprósito são definidos os seguintes objetivos específicos para este projeto PIBIC:

\begin{itemize}
\item descrever o sistema de transporte estudado;
\item identificar parêmetros para avaliação do serviço estudado;
\item compreender modelo conceitual e matemático para descrição do problema;
\item realizar levantamento de dados;
\item compreender metodologia proposta para solução do problema;
\item realizar testes da metodologia e analisar resultados;
\item propor solução para melhoria do planejamento do serviço estudado;
\item testar solução, comparando resultados de acordo com os parâmetros identificados.
\end{itemize}

\section{Metodologia}

Este projeto será desenvolvido através de 5 atividades, conforme descrito a seguir:\\

Atividade 1: descrição detalhada sobre o processo de transporte coletivo estudado, informando toda a problemática envolvida (\emph{i.e.}, modo de planejamento do transporte atual, as incovêniencias, etc.).\\

Atividade 2: identificação de parâmetros para avaliação da qualidade do transporte.\\

Atividade 3: compreensão do modelo matemático e conceitual do problema a ser desenvolvido pela orientadora.\\

Atividade 4: solução do modelo e usando a ferramenta LINGO.\\

Atividade 5: levantamento de dados e comparação dos resultados antes e após aplicação da metodologia de solução desenvolvida.\\

Atividade 6: preparação de relatórios e artigo para o CONIC.

Tais atividades serão desenvolvidas pela aluna, sob a orientação e supervisão da Profa. orientadora (autora deste projeto).

\section{Resulados esperados}

Após a aplicação da metodologia acima descrita, esperamos obter os seguintes resultados: 

\begin{itemize}
    \item descrição de um serviço do tipo \emph{Vanpooling};
    \item definição de parâmetros para avaliação deste serviço;
    \item dados sobre o processo estudado e resultados;
    \item aprofundamento do conhecimento da aluna e orientadora sobre o problema estudado e abordagem de solução utilizada;
    \item artigo apresentado no CONIC.
\end{itemize}

Os resultados destes trabalhos também serão incluídos em pelo menos um artigo que será submetido para revista.

\section{Viabilidade de execução}

O projeto será realizado \emph{in loco} e no CAA-UFPE. Assim, a aluna terá acesso garantido a toda a infraestrutura necessária para o correto desenvolvimento de seu trabalho incluindo recursos físicos (sala e mobiliário), bibliográficos e computacionais da UFPE e mais especificamente do CAA. O departamento de engenharia de produção do CAA conta atualmente com dois laboratórios de informática que disponibilizam, pelo menos, 30 computadores. O GAMOS, em especial, conta com laboratório próprio, e que atualmente dispõe de 3 computadores. Em termos de recursos bibliográficos, os pesquisadores da área de engenharia da produção contam com a biblioteca central da UFPE e as bibliotecas setoriais do CTG (Centro de Tecnologia e Geociências) e do CCEN (Centro de Ciências Exatas e da Natureza), localizadas no campus da UFPE de Recife, e com a biblioteca do próprio CAA, que possuem assinatura de alguns dos principais periódicos na área além do acesso remoto à base de dados disponíveis hoje via rede entre elas o banco de dados disponibilizados pela CAPES e pelo sciencedirect.  

\section{Cronograma de atividades do estudante}

Este projeto foi planejado para ser realizado durante o período de 1 ano, com início previsto para 2022/2023. As atividades descritas na metodologia estão projetadas para serem realizadas conforme cronograma apresentado na Tab. 1:

\begin{table}[h]
\begin{center}
\begin{tabular}[c]{||c||c|c|c|c|c|c||}
\cline {1-7}
\multirow{2}{*}{Atividade} & \multicolumn{6}{c ||}{Cronograma (bimestre)} \\ \cline {2-7}
 & $1^o$ & $2^o$ & $3^o$ & $4^o$ & $5^o$ & $6^o$ \\ \cline {1-7}
$1^a$ & xx &  &  &  &  &  \\
$2^a$ & xx &  &  &  &  &  \\
$3^a$ &  & xx & xx &  &  &  \\
$4^a$ &  &  & xx & xx & xx &   \\
$5^a$ &  &  & xx & xx & xx & xx  \\
$6^a$ &  &  &  &  &  & xx  \\ \cline {1-7}
\end{tabular}
\label{tab:Cronograma}
\caption{Cronograma planejado para o projeto.}
\end{center}
\end{table}

%\bibliographystyle{plain}
%\bibliography{bib_file}

\clearpage
\begin{thebibliography}{12}

\bibitem[\protect\citeauthoryear{Ditmore \& Deming}{2018}]{DitmoreDeming2018}
Ditmore, C.J., e Deming, D.M. (2018). 'Vanpooling and its effect on commuter stress'. {\it Research in Transportation Business \& Management}, Vol. 27, pp. 98--106.

\bibitem[\protect\citeauthoryear{Fraga}{2018}]{Fraga2018}
Fraga, T.B. (2018). 'Análise e Modelagem de Problemas de Otimização Contínua e Combinatória'. Projeto de Pesquisa registrado em 21/09/2018, e aprovado pela Pró-reitoria de Pesquisa da UFPE em 09/12/2021 (Processo SIPAC 23076.037097/2021-67).

\bibitem[\protect\citeauthoryear{Kaan \& Olinick}{2013}]{KaanOlinick2013}
Kaan, L., e Olinick, E.V. (2013). 'The Vanpool Assignment Problem: Optimization models and solution algorithms'. {\it Computers \& Industrial Engineering}, Vol. 66(1), pp. 24--40.

\bibitem[\protect\citeauthoryear{Nassar}{1986}]{Nassar1986}
Nassar, S.M. (1986). 'Simulation model for a vanpooling system'. {\it Computers \& Industrial Engineering}, Vol. 11(1-4), pp. 395--400.

\bibitem[\protect\citeauthoryear{Shared-Use Mobility Center}{}]{SMC}
Shared-Use Mobility Center. {\it What is Shared Mobility ?} Disponível em hhttps://sharedusemobilitycenter.org/what-is-shared-mobility/, último acesso em 12/07/23.

\bibitem[\protect\citeauthoryear{Wikipedia}{}]{Wikipedia}
Wikipedia. {\it Shared transport}. Disponível em https:/en.wikipedia.org/wiki/Shared\_transport, último acesso em 12/07/23.


\end{thebibliography} 

\end{document}
